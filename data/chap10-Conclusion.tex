% !TeX root = ../thesis.tex

\chapter{Conclusion and Future Work}
Chiplet architecture, is widely considered the extension of Moore’s Law and is promising to power future high-performance computing systems. In order to advance the development of chiplet architecture, I have conducted a series of studies from on-chiplet modules to chiplet-based interconnected systems, addressing several important challenges and innovating new architectures.

\textit{Chiplet Actuary} is the first quantitative cost model that introduces D2D overhead and NRE cost. It gives important insights and motivations into chiplet architectures. The \textit{scalable methodology for chiplet networks} provides a specific solution to scale out 2D-mesh-NoC-based chiplets into large-scale interconnected systems. The \textit{heterogeneous die-to-die interface} addresses the limitation of uniform interfaces and significantly improves the flexibility of chiplet-based networks. The \textit{chiplet network simulator} achieves one order of magnitude speedup over existing simulators while verifying necessary microarchitectures and maintaining cycle accuracy. The \textit{ring-road} NoC architecture addresses several major limitations of chiplet-based direct networks. The \textit{switch-less Dragonfly on wafers} eliminates costly high-radix switches of traditional supercomputer/datacenter topologies by utilizing wafer-scale integration while increasing local throughput and maintaining global throughput, promising to power future large-scale
supercomputers. The \textit{RailX} is developed from \textit{switch-less Dragonfly wafers} and is further optimized for large-scale AI training workloads. By using wafer-scale integration and optical circuit switching, \textit{RailX} significantly improves the cost-efficiency of AI systems. These studies revealed three important insights into the development of more scalable computer architectures:
    \begin{itemize}
        \item \textbf{Advances in hardware technologies make a real difference.} Just as Moore's Law has driven chip performance advances over the past few decades, advances in computer architecture cannot be made without advances in hardware technologies. In recent years, many emerging hardware technologies, including 2D/2.5D/3D integration, high-speed wireline, co-packaged optics, and optical switching, have made great progress. These technologies allow chips to be integrated and interconnected with high bandwidth and low latency, promising to inspire new architectures. 
        \item \textbf{Domain-specific architectures maximize benefits.} Domain-specific architectures, especially AI accelerators, have had great success in computing chips. Similarly, workloads can also inspire innovations in interconnection networks. Existing network architectures are designed for general-purpose workloads. Thus, they are not cost-effective for extremely specific workloads such as hyper-scale LLM training. Workload-network co-design promises to bring significant improvement to existing architectures.
        \item \textbf{Networking from a comprehensive view is the key.} Previously, on-chip networks and system-level networks were designed separately, resulting in suboptimal performance. To design a more scalable network architecture, the entire design space, including on-chip networks, on-package networks, inter-package networks, routing, flow control, etc., must be considered as a whole. Breaking through the boundary of system hierarchies can reveal more potential optimization spaces.
    \end{itemize}
\textbf{Putting all these together is the path towards further scaling.}

\section{Feature Work}
The advancements in chiplet architectures and interconnection networks present numerous opportunities for further exploration. Future research directions include:
\begin{itemize}
    \item Advanced Integration Technologies: Investigating emerging 2.5D/3D integration techniques, such as hybrid bonding and monolithic 3D stacking, to further reduce D2D overhead and improve energy efficiency. Going beyond basing on the 2D-mesh architecture can bring potentially huge benefits. The integration of co-packaged optics (CPO) and silicon photonics could also enable ultra-low-latency, high-bandwidth interconnects for wafer-scale systems. 
    \item Workload-Aware Network Optimization: Extending domain-specific co-design methodologies to other high-performance computing (HPC) and AI workloads, such as scientific simulations or real-time inference. Dynamic reconfigurability (e.g., via optical circuit switching or programmable NoCs) could adapt networks to workload phases, improving throughput and energy efficiency.
    \item Reliability: Addressing challenges in fault tolerance and thermal management in disaggregated chiplet systems. Redundant interconnects, runtime monitoring, and reliable D2D communication protocols will be critical for deployment in large-scale production environments.
\end{itemize}
Pursuing these directions will further explore the potential for scaling, and the deployment difficulties can be addressed, paving the way for next-generation computing systems.
 