% !TeX root = ../thesis.tex

\begin{resume}

  \section*{个人简历}

  1997 年 9 月 16 日出生于浙江湖州。

  2016 年 9 月考入上海交通大学电子信息与电气工程学院,2020 年 6 月本科毕业并获得信息工程工学学士学位,同时获得数学与应用数学第二学科理学学士学位和致远工学荣誉学士学位。

  2020 年 8 月免试进入清华大学交叉信息研究院攻读计算机科学与技术博士。


  \section*{在学期间完成的相关学术成果}

  \subsection*{学术论文}

  \begin{achievements}
    \item \textbf{Yinxiao Feng}, Wei Li, and Anonymous. Ring Road: A Scalable Polar-Coordinate-Based 2D Network-on-Chip Architecture[C]. International Symposium on Microarchitecture (MICRO), 2024.
    \item \textbf{Yinxiao Feng} and Anonymous. Switch-Less Dragonfly on Wafers: A Scalable Interconnection Architecture Based on Wafer-Scale Integration[C]. International Conference for High-Performance Computing, Networking, Storage, and Analysis (SC), 2024. 
    \item \textbf{Yinxiao Feng}, Yuchen Wei, Dong Xiang, and Anonymous. Evaluating Chiplet-Based Large-Scale Interconnection Networks via Cycle-Accurate Packet-Parallel Simulation[C]. USENIX Annual Technical Conference (ATC), 2024.
    \item \textbf{Yinxiao Feng}, Dong Xiang, and Anonymous. 2023. Heterogeneous Die-to-Die Interfaces: Enabling More Flexible Chiplet Interconnection Systems[C]. International Symposium on Microarchitecture (MICRO), 2023.
    \item \textbf{Yinxiao Feng}, Dong Xiang, and Anonymous. A Scalable Methodology for Designing Efficient Interconnection Network of Chiplets[C]. International Symposium on High-Performance Computer Architecture (HPCA), 2023.
    \item \textbf{Yinxiao Feng} and Anonymous. Chiplet Actuary: A Quantitative Cost Model and Multi-Chiplet Architecture Exploration[C]. Design Automation Conference (DAC), 2022 
  \end{achievements}

\end{resume}
